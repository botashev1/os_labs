
\documentclass[pdf, unicode, 12pt, a4paper,oneside,fleqn]{article}
\usepackage{graphicx}
\usepackage{upquote}
\graphicspath{{img/}}
\usepackage{log-style}
\begin{document}

\begin{titlepage}
    \begin{center}
        \bfseries

        {\Large Московский авиационный институт\\ (национальный исследовательский университет)}
        
        \vspace{48pt}
        
        {\large Факультет информационных технологий и прикладной математики}
        
        \vspace{36pt}
        
        {\large Кафедра вычислительной математики и~программирования}
        
        \vspace{48pt}
        
        Лабораторная работа \textnumero 6-8 по курсу \enquote{Операционные системы}

        \vspace{48pt}

        Управлении серверами сообщений. Применение отложенных вычислений. Интеграция программных систем друг с другом
        
    \end{center}
    
    \vspace{125pt}
    
    \begin{flushright}
    \begin{tabular}{rl}
    Студент: & А.\,Р. Боташев \\
    Преподаватель: & Е.\,С. Миронов \\
    Группа: & М8О-201Б-21 \\
    Вариант: & 23 \\
    Дата: & \\
    Оценка: & \\
    Подпись: & \\
    \end{tabular}
    \end{flushright}
    
    \vfill
    
    \begin{center}
    \bfseries
    Москва, \the\year
    \end{center}
\end{titlepage}
    
\pagebreak

\section{Постановка задачи}

Реализовать распределенную систему по асинхронной обработке запросов. 
В данной распределенной системе должно существовать 2 вида узлов: 
«управляющий» и «вычислительный». Необходимо объединить данные узлы в 
соответствии с той топологией, которая определена вариантом. Связь между
узлами необходимо осуществить при помощи технологии очередей сообщений.
Также в данной системе необходимо предусмотреть проверку доступности
узлов в соответствии с вариантом. При убийстве («kill -9») любого
вычислительного узла система должна пытаться максимально сохранять
свою работоспособность, а именно все дочерние узлы убитого узла могут стать
недоступными, но родительские узлы должны сохранить свою работоспособность.

Управляющий узел отвечает за ввод команд от пользователя и отправку этих команд
на вычислительные узлы.

Топология \--- дерево общего вида, проверка \--- доступности ping id, вычислительная
программа \--- поиск подстоки в строке.

\section{Общие сведения о программе}
Программа написанна на языке Си в Unix подобной операционной системе на базе ядра Linux.
В программе используется очередь сообщений ZeroMQ.

Программа поддерживает следующие команды:
\begin{enumerate}
    \item create [id] [parent] \--- создать узел с id [id], родителем которого является узел с id [parent].
    \item remove [id] \--- удаляет узел с данным id.
    \item exec [id] [string] [pattern] \--- запускает на узле [id] поиск подстроки [pattern] в строке [string].
    \item ping [id] \--- проверка доступности узда [id].
\end{enumerate}

Выход из программы происзодит при окончании ввода, то есть при нажатии CTRL+D.

\section{Общий метод и алгоритм решения}

В программе используется тип соединения Request-Response. Узлы прередают
друг другу информацию при помощи очереди сообщений.

Все функции описанны либо для самого целевого узла либо его родителя. А для 
доставки сообщений создан метод Send(). Этот методо проверяет на доступность всех детей,
если они доступны то отправляет им сообщение. В случае если детей нет, то он возвращает родителю
ошибку об не нахождении узла. Когда родитель получает ответы от всех детей он 
отправляет на уровень выше либо единственный отличающийся от ошибки поиска, либо саму ошибку поиска.
И так пока ответ не достигнет отправителя.

Если команда обращается к несуществующему узлу, то мы сразу возвращаем ошибку. Для этого у
нас есть множество созданных узлов.

При создании мы просо форкаем родитель и передаем ребенку данные для связи с ним.

При удалении же мы передаем все детям также сигнал об удалении и рекурсивно 
формируем список всех своих потомков. В который по итогу добавляем себя и отправляем родителю.
По итогу клиент удаляет все элименты этого списка из множества созданных узлов.

Для проверки доступности мы посылаем ребенку сообщение, и если за 3 секунды не получаем ответа, то считаем узел не доступным.

Поиск подстроки в строке происходит с помощью наивного алгорима поиска из стандарной библиотеки.

\section{Листинг программы}

{\large\textbf{net\_func.hpp}}

\begin{lstlisting}[language=C++]
#pragma once

#include <iostream>
#include <zmq.hpp>
#include <sstream>
#include <string>

namespace my_net{
    #define MY_PORT 4040
    #define MY_IP "tcp://127.0.0.1:"

    int bind(zmq::socket_t *socket, int id){
        int port  = MY_PORT + id;
        while(true){
            std::string adress = MY_IP  + std::to_string(port);
            try{
                socket->bind(adress);
                break;
            } catch(...){
                port++;
            }    
        }
        return port;
    }

    void connect(zmq::socket_t *socket, int port){
        std::string adress = MY_IP + std::to_string(port);
        socket->connect(adress);
    }

    void unbind(zmq::socket_t *socket, int port) {
        std::string adress = MY_IP + std::to_string(port);
        socket->unbind(adress);
    }

    void disconnect(zmq::socket_t *socket, int port) {
        std::string adress = MY_IP + std::to_string(port);
        socket->disconnect(adress);
    }

    void send_message(zmq::socket_t *socket, const std::string msg) {
        zmq::message_t message(msg.size());
        memcpy(message.data(), msg.c_str(), msg.size());
        try{
            socket->send(message);
        }catch(...){}
    }

    std::string reseave(zmq::socket_t *socket){
        zmq::message_t message;
        bool success = true;
        try{
            socket->recv(&message,0);
        }catch(...){
            success = false;
        }
        if(!success || message.size() == 0){;
            throw -1;
        }
        std::string str(static_cast<char*>(message.data()), message.size());
        return str;
    }
}
\end{lstlisting}

{\large\textbf{node.hpp}}

\begin{lstlisting}[language=C++]
#include <iostream>
#include "net_func.hpp"
#include <sstream>
#include <unordered_map>
#include "unistd.h"


class Node{
private:
    zmq::context_t context;
public:
    std::unordered_map<int,zmq::socket_t*> children;
    std::unordered_map<int,int> children_port;
    zmq::socket_t parent;
    int parent_port;

    int id;
    Node(int _id , int _parent_port = -1):   id(_id),
                                            parent(context,ZMQ_REP),
                                            parent_port(_parent_port){
        if(_id != -1){
            my_net::connect(&parent,_parent_port);
        }
    }

    std::string Ping_child(int _id){
        std::string ans = "Ok: 0";
        ans = "Ok: 0";
        if(_id == id){
            ans = "Ok: 1";
            return ans;
        } else if(children.find(_id) != children.end()){
            std::string msg = "ping " + std::to_string(_id);
            my_net::send_message(children[_id],msg);
            try{
                msg  = my_net::reseave(children[_id]);
                if(msg == "Ok: 1")
                    ans = msg;
            } catch(int){}
            return ans;
        }else{
            return ans;
        }
    } 

    std::string Create_child(int child_id,std::string program_path){
        std::string program_name = program_path.substr(program_path.find_last_of("/") + 1);
        children[child_id] = new zmq::socket_t(context,ZMQ_REQ);
        
        int new_port = my_net::bind(children[child_id],child_id);
        children_port[child_id] = new_port;
        int pid = fork();
        
        if(pid == 0){
            execl(program_path.c_str(), program_name.c_str(), std::to_string(child_id).c_str() , std::to_string(new_port).c_str() ,(char*)NULL);
        }else{
            std::string child_pid;
            try{
                children[child_id]->setsockopt(ZMQ_SNDTIMEO,3000);
                my_net::send_message(children[child_id],"pid");
                child_pid = my_net::reseave(children[child_id]);
            } catch(int){
                child_pid = "Error: can't connect to child";
            }
            return "Ok: " + child_pid;
        }
    }

    std::string Pid(){
        return std::to_string(getpid());
    }

    std::string Send(std::string str,int _id){
        if(children.size() == 0){
            return "Error: now find";
        }else if(children.find(_id) != children.end()){
            if(Ping_child(_id) == "Ok: 1"){
                my_net::send_message(children[_id],str);
                std::string ans;
                try{
                    ans = my_net::reseave(children[_id]);
                } catch(int){
                    ans = "Error: now find";
                }
                return ans;
            }
        }else{
            std::string ans = "Error: not find";
            for(auto& child: children ){
                if(Ping_child(child.first) == "Ok: 1"){
                    std::string msg = "send " + std::to_string(_id) + " " + str;
                    my_net::send_message(children[child.first],msg);
                    try{
                        msg = my_net::reseave(children[child.first]);
                    } catch(int){
                        msg = "Error: not find";
                    }
                    if(msg != "Error: not find"){
                        ans = msg;
                    }
                }
            }
            return ans;
        }
    }

    std::string Remove(){
        std::string ans;
        if(children.size() > 0){
            for(auto& child: children ){
                if(Ping_child(child.first) == "Ok: 1"){
                    std::string msg = "remove";
                    my_net::send_message(children[child.first],msg);
                    try{
                        msg = my_net::reseave(children[child.first]);
                        if(ans.size() > 0)
                            ans = ans + " " + msg;
                        else
                            ans =  msg;
                    } catch(int){}
                }
                my_net::unbind(children[child.first], children_port[child.first]);
                children[child.first]->close();
            }
            children.clear();
            children_port.clear();
        }
        return ans;
    }
};       
\end{lstlisting}

{\large\textbf{worker.cpp}}

\begin{lstlisting}[language=C++]
#include "node.hpp"
#include "net_func.hpp"
#include <fstream>
#include <vector>
#include <signal.h>

int my_id = 0;

void Log(std::string str){
    std::string f = std::to_string(my_id) + ".txt";
    std::ofstream fout(f,std::ios_base::app);
    fout << str;
    fout.close();
}

int main(int argc, char **argv){
    if(argc != 3){
        return -1;
    }
    
    Node me(atoi(argv[1]),atoi(argv[2]));
    std::string prog_path = "./worker";
    while(1){
        std::string message;
        std::string command = " ";
        message = my_net::reseave(&(me.parent));
        std::istringstream  request(message);
        request >> command;

        
        if(command == "create"){
            int id_child, id_parent;
            request >> id_child;
            std::string ans = me.Create_child(id_child, prog_path);
            my_net::send_message(&me.parent,ans);
        } else if(command == "pid"){
            std::string ans = me.Pid();
            my_net::send_message(&me.parent,ans);
        } else if(command == "ping"){
            int id_child;
            request >> id_child;
            std::string ans = me.Ping_child(id_child);
            my_net::send_message(&me.parent,ans);
        } else if(command == "send"){
            int id;
            request >> id;
            std::string str;
            getline(request, str);
            str.erase(0,1);
            std::string ans;
            ans = me.Send(str,id);
            my_net::send_message(&me.parent,ans);
        } else if(command == "exec"){
            std::string str;
            std::string pattern;
            request >> str >> pattern;
            std::vector<int> answers;
            int start = 0;
            while(str.find(pattern,start) != -1){
                start = str.find(pattern,start);
                answers.push_back(start);
                start++;
            }
            std::string to_send;
            if(answers.size() == 0){
                to_send = "-1";
            }else{
                to_send = std::to_string(answers[0]);
                for(int i = 1; i < answers.size();++i){
                    to_send = to_send + ";" + std::to_string(answers[i]);
                }
            }
            to_send = "Ok:" + std::to_string(me.id) + ":" + to_send;
            my_net::send_message(&me.parent,to_send);
        } else if(command == "remove"){
            std::string ans = me.Remove();
            ans = std::to_string(me.id) + " " + ans;
            my_net::send_message(&me.parent, ans);
            my_net::disconnect(&me.parent, me.parent_port);
            me.parent.close();
            break;
        }
    }
    sleep(1);
    return 0;
}
\end{lstlisting}

{\large\textbf{client.cpp}}

\begin{lstlisting}[language=C++]
#include "node.hpp"
#include "net_func.hpp"
#include "set"
#include <signal.h>



int main(){
    std::set<int> all_nodes;
    //std::set<int> not_availvable;
    std::string prog_path = "./worker";
    Node me(-1);
    all_nodes.insert(-1);
    std::string command;
    while(std::cin >> command){
        if(command == "create"){
            int id_child, id_parent;
            std::cin >> id_child >> id_parent;
            if(all_nodes.find(id_child) != all_nodes.end()){
                std::cout << "Error: Already exists" << std::endl;
            } else if(all_nodes.find(id_parent) == all_nodes.end()){
                std::cout << "Error: Parent not found" << std::endl;
            }else if(id_parent == me.id){
                std::string ans = me.Create_child(id_child, prog_path);
                std::cout << ans << std::endl;
                all_nodes.insert(id_child);
            } else{
                std::string str = "create " + std::to_string(id_child);
                std::string ans = me.Send(str, id_parent);
                std::cout << ans << std::endl;
                all_nodes.insert(id_child);
            }   
        } else if(command == "ping"){
            int id_child;
            std::cin >> id_child;
            if(all_nodes.find(id_child) == all_nodes.end()){
                std::cout << "Error: Not found" << std::endl;
            }else if(me.children.find(id_child) != me.children.end()){
                std::string ans = me.Ping_child(id_child);
                std::cout << ans << std::endl;
            }else{
                std::string str = "ping " + std::to_string(id_child);
                std::string ans = me.Send(str, id_child);
                if(ans == "Error: not find"){
                    ans = "Ok: 0";
                }
                std::cout << ans << std::endl;
            }
        }else if(command == "exec"){
            int id;
            std::string str;
            std::string pattern;
            std::cin >> id >> str >> pattern;
            std::string msg = "exec " + str + " " + pattern;
            if(all_nodes.find(id) == all_nodes.end()){
                std::cout << "Error: Not found" << std::endl;
            }else{
                std::string ans = me.Send(msg,id);
                std::cout << ans << std::endl;
            }
            
        }else if(command == "remove"){
            int id;
            std::cin >> id;
            std::string msg = "remove";
            if(all_nodes.find(id) == all_nodes.end()){
                std::cout << "Error: Not found" << std::endl;
            }else{
                std::string ans = me.Send(msg,id);
                if(ans != "Error: not find"){
                    std::istringstream ids(ans);
                    int tmp;
                    while(ids >> tmp){
                        all_nodes.erase(tmp);
                    }
                    ans = "Ok";
                    if(me.children.find(id) != me.children.end()){
                        my_net::unbind(me.children[id],me.children_port[id]);
                        me.children[id]->close();
                        me.children.erase(id);
                        me.children_port.erase(id);
                    }
                }
                std::cout << ans << std::endl;
            }
        }
    }
    me.Remove();
    return 0;
}
\end{lstlisting}


\pagebreak

\section{Демонстрация работы программы}

\begin{alltt}
    botashev@botashev-laptop:~/ClionProjects/os_labs/lab6-8$ ./client 
create 2 -1
Ok: 49589
create 3 2
Ok: 49593
create 4 3
Ok: 49597
create 5 3
Ok: 49601
ping 5
Ok: 1
exec 3
abracadabra
abra
Ok:3:0;7
remove 5
Ok
ping 5
Error: Not found
remove 3
ping 3
Ok
Error: Not found
remove 4
Error: Not found
\end{alltt}

\pagebreak

\section{Вывод}
Наряду с каналами и отображаемыми файлами для передачи данных применяются очереди сообщений.

Данная лабораторная работа была направлена на изучение технологий очередей сообщений 
и сокетов. Я ознакомился и изучил их, реализовав сеть с заданной топологей. При создании я использовал
библиотеку для передачи сообщений ZeroMQ. ZeroMQ предоставляет простой интерфейс для передачи сообщений и 
составления топологий разных типов.

Полученные навыки можно применить во многих сферах, так как большинство программ сейчас взаимодействуют по сети.
При этом ZeroMQ можно применить и для внутреннего межпроцессорного взаимодействия.
\end{document}
