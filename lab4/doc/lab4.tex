
\documentclass[pdf, unicode, 12pt, a4paper,oneside,fleqn]{article}
\usepackage{graphicx}
\graphicspath{{img/}}
\usepackage{log-style}
\begin{document}

\begin{titlepage}
    \begin{center}
        \bfseries

        {\Large Московский авиационный институт\\ (национальный исследовательский университет)}
        
        \vspace{48pt}
        
        {\large Факультет информационных технологий и прикладной математики}
        
        \vspace{36pt}
        
        {\large Кафедра вычислительной математики и~программирования}
        
        \vspace{48pt}
        
        Лабораторная работа \textnumero 4 по курсу \enquote{Операционные системы}

        \vspace{48pt}

        Освоение принципов работы с файловыми системами. 
        
        Обеспечение обмена данных между процессами посредством технологии <<File mapping>>.
    \end{center}
    
    \vspace{125pt}
    
    \begin{flushright}
    \begin{tabular}{rl}
    Студент: & А.\,Р. Боташев \\
    Преподаватель: & Е.\,С. Миронов \\
    Группа: & М8О-201Б-21 \\
    Вариант: & 12 \\
    Дата: & \\
    Оценка: & \\
    Подпись: & \\
    \end{tabular}
    \end{flushright}
    
    \vfill
    
    \begin{center}
    \bfseries
    Москва, \the\year
    \end{center}
\end{titlepage}
    
\pagebreak

\section{Постановка задачи}

Составить и отладить программу на языке Си, осуществляющую работу с процессами 
и взаимодействие между ними в одной из двух операционных систем. В результате работы 
программа (основной процесс) должен создать для решения задачи один или несколько 
дочерних процессов. Взаимодействие между процессами осуществляется через системные 
сигналы/события и/или через отображаемые файлы (memory-mapped files).

Необходимо обрабатывать системные ошибки, которые могут возникнуть в результате работы.

Родительский процесс создает два дочерних процесса. Перенаправление стандартных потоков
Child1 и Child2 можно «соединить» между собой
дополнительным каналом. Родительский и дочерний процесс должны быть представлены
разными программами.

Родительский процесс принимает от пользователя строки произвольной длины и пересылает их в
pipe1. Процесс child1 и child2 производят работу над строками. Child2 пересылает результат своей
работы родительскому процессу. Родительский процесс полученный результат выводит в
стандартный поток вывода.


Child1 переводит строки в верхний регистр. Child2 убирает все задвоенные пробелы.

\section{Сведения о программе}

Программа написанна на Си в Unix подобной операционной системе на базе ядра Linux.
При компиляции следует линковать библиотеки -lpthread и -lrt.
В программе создаются дочерние процессы, данные в которые передаются с помощью shared memory.

При запуске программы пользователь вводит строки в стандартный поток ввода. Программа создает два дочерних процесса для преобразования введенных строк.

По завершении работы программа выводит в стандартный поток вывода введенные строки в верхнем регистре, удалив все задвоенные пробелы


\section{Общий метод и алгоритм решения}

Родительский процесс создает первый дочерний процесс, передав строки, полученные от пользователя. Затем родительский процесс создает второй дочерний процесс.

Первый дочерний прочесс принимает строки и приводит все символы в верхний регистр, после чего передавая полученные строки во второй дочерний процесс

Второй дочерний процесс принимает строки через pipe3, после чего удаляет все задвоенные пробелы и передает полученные строки родительскому процессу

Результирующие строки родительский процесс считывает от второго дочернего процесса

Все межпроцессорные взаимодействия реализованы через shared memory object. Для синхронизации работы процессов используются семафоры

\section{Листинг программы}

{\large\textbf{main.cpp}}

\begin{lstlisting}[language=C++]
#include "include/parent.h"
#include <vector>

int main() {
    std::vector <std::string> input;
    std::string s;

    while (getline(std::cin, s)) {
        input.push_back(s);
    }

    std::vector <std::string> output = ParentRoutine("4child1", "4child2", input);

    for (const auto &res : output){
        std::cout << res << std::endl;
    }
    return 0;
}

\end{lstlisting}

{\large\textbf{parent.cpp}}

\begin{lstlisting}[language=C++]
#include "parent.h"
#include "utils.h"
#include <sys/mman.h>
#include <unistd.h>


constexpr auto FIRST_SHM_NAME = "shared_memory_first"; // from parent to child1
constexpr auto SECOND_SHM_NAME = "shared_memory_second"; // from child2 to parent
constexpr auto THIRD_SHM_NAME = "third_shared_memory"; // from child1 to child2
constexpr auto FIRST_SEMAP = "first_semaphore";
constexpr auto SECOND_SEMAP = "second_semaphore";
constexpr auto THIRD_SEMAP = "third_semaphore";


std::vector<std::string> ParentRoutine(char const *pathToChild1, char const *pathToChild2,
                                       const std::vector<std::string> &input) {

    std::vector<std::string> output;

    int sfd1, semFd1;
    createShm(sfd1, semFd1, FIRST_SHM_NAME, FIRST_SEMAP);
    makeFtruncateShm(sfd1, semFd1);
    sem_t *sem1 = nullptr;
    makeMmap((void **) &sem1, PROT_WRITE | PROT_READ, MAP_SHARED, semFd1);
    sem_init(sem1, 1, 0);


    int sfd2, semFd2;
    createShm(sfd2, semFd2, SECOND_SHM_NAME, SECOND_SEMAP);
    makeFtruncateShm(sfd2, semFd2);
    sem_t *sem2 = nullptr;
    makeMmap((void **) &sem2, PROT_WRITE | PROT_READ, MAP_SHARED, semFd2);
    sem_init(sem2, 1, 0);


    int pid = fork();


    if (pid == 0) { // child1
        if (execl(pathToChild1, FIRST_SHM_NAME, FIRST_SEMAP,
                  THIRD_SHM_NAME, THIRD_SEMAP, nullptr) == -1) {
            GetExecError(pathToChild1);
        }
    } else if (pid == -1) {
        GetForkError();
    } else {
        char *ptr;
        makeMmap((void **) &ptr, PROT_READ | PROT_WRITE, MAP_SHARED, sfd1);
        for (const std::string &s: input) {
            sprintf((char *) ptr, "%s", s.c_str());
            ptr += s.size() + 1;
            sem_post(sem1);
        }
        sprintf((char *) ptr, "%s", "");
        sem_post(sem1);

        int sfd3, semFd3;
        createShm(sfd3, semFd3, THIRD_SHM_NAME, THIRD_SEMAP);
        makeFtruncateShm(sfd3, semFd3);
        sem_t *sem3 = nullptr;
        makeMmap((void **) &sem3, PROT_WRITE | PROT_READ, MAP_SHARED, semFd3);
        sem_init(sem3, 1, 0);


        pid = fork();

        if (pid == 0) { // child2
            if (execl(pathToChild2, THIRD_SHM_NAME, THIRD_SEMAP,
                      SECOND_SHM_NAME, SECOND_SEMAP, nullptr) == -1) {
                GetExecError(pathToChild2);
            }
        } else if (pid == -1) {
            GetForkError();
        } else { // parent
            char *ptr2;
            makeMmap((void **) &ptr2, PROT_READ | PROT_WRITE, MAP_SHARED, sfd2);

            while (true) {
                sem_wait(sem2);
                std::string s = std::string(ptr2);
                ptr2 += s.size() + 1;
                if (s.empty()) {
                    break;
                }
                output.push_back(s);
            }


            makeSemDestroy(sem1);
            makeMunmap(sem1);


            makeSemDestroy(sem2);
            makeMunmap(sem2);

            makeShmUnlink(FIRST_SHM_NAME);
            makeShmUnlink(SECOND_SHM_NAME);
            makeShmUnlink(FIRST_SEMAP);
            makeShmUnlink(SECOND_SEMAP);
        }
    }
    return output;
}

\end{lstlisting}

{\large\textbf{utils.cpp}}

\begin{lstlisting}[language=C++]
#include "utils.h"
#include <sys/mman.h>
#include <semaphore.h>
#include <fcntl.h>
#include <unistd.h>


void makeSharedMemoryOpen(int &sfd, std::string name, int oflag, mode_t mode) {
    if ((sfd = shm_open(name.c_str(), oflag, mode)) == -1) {
        std::cout << "Shm_open error" << std::endl;
        exit(EXIT_FAILURE);
    }
}

void makeMmap(void **var, int prot, int flags, int fd) {
    *var = mmap(nullptr, getpagesize(), prot, flags, fd, 0);
    if (var == MAP_FAILED) {
        std::cout << "Mmap error" << std::endl;
        exit(EXIT_FAILURE);
    }
}

void makeSemDestroy(sem_t *sem) {
    if (sem_destroy(sem) == -1) {
        std::cout << "Sem_destroy error" << std::endl;
        exit(EXIT_FAILURE);
    }
}

void makeMunmap(void *addr) {
    if (munmap(addr, getpagesize()) == -1) {
        std::cout << "Munmap error" << std::endl;
        exit(EXIT_FAILURE);
    }
}

void makeShmUnlink(std::string name) {
    if (shm_unlink(name.c_str()) == -1) {
        std::cout << "Shm_unlink error" << std::endl;
        exit(EXIT_FAILURE);
    }
}

void createShm(int &sfd, int &semInFd, const std::string &shmName,
               const std::string &semap) {
    makeSharedMemoryOpen(sfd, shmName, O_CREAT | O_RDWR, S_IRWXU);
    makeSharedMemoryOpen(semInFd, semap, O_CREAT | O_RDWR, S_IRWXU);
}

void makeFtruncateShm(int &sfd, int &semInFd){
    ftruncate(sfd, getpagesize());
    ftruncate(semInFd, getpagesize());
}

void GetForkError() {
    std::cout << "fork error" << std::endl;
    exit(EXIT_FAILURE);
}

void GetExecError(std::string const &executableFile) {
    std::cout << "Exec \"" << executableFile << "\" error." << std::endl;
}

\end{lstlisting}


{\large\textbf{child1.cpp}}

\begin{lstlisting}[language=C++]
#include "utils.h"
#include <sys/mman.h>
#include <unistd.h>
#include <fcntl.h>

int main(int argc, char *argv[]) {
    if (argc != 4) {
        std::cout << "Invalid arguments 1.\n";
        exit(EXIT_FAILURE);
    }

    int readFd, semInFd;
    makeSharedMemoryOpen(readFd, argv[0], O_CREAT | O_RDWR, S_IRWXU);
    makeSharedMemoryOpen(semInFd, argv[1], O_CREAT | O_RDWR, S_IRWXU);

    int writeFd = 0, semOutFd = 0;
    makeSharedMemoryOpen(writeFd, argv[2], O_CREAT | O_RDWR, S_IRWXU);
    makeSharedMemoryOpen(semOutFd, argv[3], O_CREAT | O_RDWR, S_IRWXU);


    char *input, *output;
    sem_t *semInput, *semOutput;
    makeMmap((void **) &input, PROT_READ | PROT_WRITE, MAP_SHARED, readFd);
    makeMmap((void **) &output, PROT_READ | PROT_WRITE, MAP_SHARED, writeFd);
    makeMmap((void **) &semInput, PROT_READ | PROT_WRITE, MAP_SHARED, semInFd);
    makeMmap((void **) &semOutput, PROT_READ | PROT_WRITE, MAP_SHARED, semOutFd);

    char *ptrIn = input, *ptrOut = output;


    while (true) {
        sem_wait(semInput);
        std::string s = std::string(ptrIn);
        ptrIn += s.size() + 1;
        if (s.empty()) {
            break;
        }
        for (char &ch: s) {
            ch = toupper(ch);
        }

        sprintf((char *) ptrOut, "%s", s.c_str());
        ptrOut += s.size() + 1;
        sem_post(semOutput);
    }
    sprintf((char *) ptrOut, "%s", "");
    sem_post(semOutput);

    makeMunmap(input);
    makeMunmap(output);
    makeMunmap(semInput);
    makeMunmap(semOutput);

    return 0;
}
\end{lstlisting}

{\large\textbf{child2.cpp}}

\begin{lstlisting}[language=C++]
#include "utils.h"
#include <sys/mman.h>
#include <unistd.h>
#include <fcntl.h>


int main(int argc, char *argv[]) {
    if (argc != 4) {
        std::cout << "Invalid arguments 2.\n";
        exit(EXIT_FAILURE);
    }

    int readFd, semInFd;
    makeSharedMemoryOpen(readFd, argv[0], O_CREAT | O_RDWR, S_IRWXU);
    makeSharedMemoryOpen(semInFd, argv[1], O_CREAT | O_RDWR, S_IRWXU);

    int writeFd, semOutFd;
    makeSharedMemoryOpen(writeFd, argv[2], O_CREAT | O_RDWR, S_IRWXU);
    makeSharedMemoryOpen(semOutFd, argv[3], O_CREAT | O_RDWR, S_IRWXU);


    char *input, *output;
    sem_t *semInput, *semOutput;
    makeMmap((void **) &input, PROT_READ | PROT_WRITE, MAP_SHARED, readFd);
    makeMmap((void **) &output, PROT_READ | PROT_WRITE, MAP_SHARED, writeFd);
    makeMmap((void **) &semInput, PROT_READ | PROT_WRITE, MAP_SHARED, semInFd);
    makeMmap((void **) &semOutput, PROT_READ | PROT_WRITE, MAP_SHARED, semOutFd);



    char *ptrIn = input, *ptrOut = output;

    while (true) {
        sem_wait(semInput);
        std::string s = std::string(ptrIn);
        ptrIn += s.size() + 1;
        if (s.empty() || s == " ") {
            break;
        }
        int j = 0;
        char lastCh = '\0';
        for (size_t i = 0; i < s.size(); i++){
            if (lastCh != ' ' || s[i] != ' '){
                s[j] = s[i];
                j++;
            }
            lastCh = s[i];
        }

        std::string res;
        for (int i = 0; i < j; i++) {
            res += s[i];
        }

        sprintf((char *) ptrOut, "%s", res.c_str());
        ptrOut += res.size() + 1;
        sem_post(semOutput);
    }

    sprintf((char *) ptrOut, "%s", "");
    sem_post(semOutput);

    makeMunmap(input);
    makeMunmap(output);
    makeMunmap(semInput);
    makeMunmap(semOutput);
    return 0;
}
\end{lstlisting}

\section{Демонстрация работы программы}

\begin{verbatim}
botashev@botashev-laptop:~/ClionProjects/os_labs/tests$ cat lab4_test.cpp 
#include <gtest/gtest.h>

#include <array>
#include <memory>
#include <parent.h>
#include <vector>


TEST(FirstLabTests, SimpleTest) {

    constexpr int inputSize = 3;

    std::array< std::vector<std::string>, inputSize> input;
    input[0] = {
            "abcabc",
            "qwerty qwerty",
            "A n O t H e R             TeSt",
            "oNe1 Two2  thr3ee   5fiVe     Ei8ght        13thiRTEEN             ...",
            "2 + 2 = 4",
            "0123456789 abcdefghijklmnopqrstuvwxyz"
    };
    input[1] = {
            "second test",
            "1234567890/.,'][",
            ".            .             .                         ...............",
            "!?+-*/_;",
    };
    input[2] = {
            "AAAAAAAAAAAAAAAAAAAAAAAAAAAAA"
    };

    std::array< std::vector<std::string>, inputSize> expectedOutput;
    expectedOutput[0] = {
            "ABCABC",
            "QWERTY QWERTY",
            "A N O T H E R TEST",
            "ONE1 TWO2 THR3EE 5FIVE EI8GHT 13THIRTEEN ...",
            "2 + 2 = 4",
            "0123456789 ABCDEFGHIJKLMNOPQRSTUVWXYZ"
    };
    expectedOutput[1] = {
            "SECOND TEST",
            "1234567890/.,'][",
            ". . . ...............",
            "!?+-*/_;",
    };
    expectedOutput[2] = {
            "AAAAAAAAAAAAAAAAAAAAAAAAAAAAA"
    };


    for (int i = 0; i < inputSize; i++) {
        auto result = ParentRoutine(
                "/home/botashev/ClionProjects/os_labs/cmake-build-debug/lab4/4child1",
                "/home/botashev/ClionProjects/os_labs/cmake-build-debug/lab4/4child2",
                input[i]);
        EXPECT_EQ(result, expectedOutput[i]);
    }
}
botashev@botashev-laptop:~/ClionProjects/os_labs/tests$ ./../cmake-build-debug/tests/lab4_test 
Running main() from /home/botashev/ClionProjects/os_labs/cmake-build-debug/_deps/googletest-src/googletest/src/gtest_main.cc
[==========] Running 1 test from 1 test suite.
[----------] Global test environment set-up.
[----------] 1 test from FirstLabTests
[ RUN      ] FirstLabTests.SimpleTest
[       OK ] FirstLabTests.SimpleTest (5 ms)
[----------] 1 test from FirstLabTests (5 ms total)

[----------] Global test environment tear-down
[==========] 1 test from 1 test suite ran. (5 ms total)
[  PASSED  ] 1 test.
\end{verbatim}

\pagebreak

\section{Вывод}

Взаимодействие между процессами можно организовать при помощи каналов, сокетов и отображаемых
файлов. В данной лабораторной работе был изучен и применен механизм межпроцессорного взаимодействия 
\--- file mapping. Файл отображается на оперативную память таким образом, что мы можем взаимодействовать
с ним как с массивом.

Благодаря этому вместо медленных запросов на чтение и запись мы выполняем отображение файла в
ОЗУ и получаем произвольный доступ за O(1). Из-за этого при использовании этой технологии межпроцессорного
взаимодействия мы можем получить ускорении работы программы, в сравнении, с использованием каналов.

Из недостатков данного метода можно выделить то, что дочерние процессы обязательно должны знать
имя отображаемого файла и также самостоятельно выполнить отображение.
\end{document}

