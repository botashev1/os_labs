
\documentclass[pdf, unicode, 12pt, a4paper,oneside,fleqn]{article}
\usepackage{graphicx}
\graphicspath{{img/}}
\usepackage{log-style}
\begin{document}

\begin{titlepage}
    \begin{center}
        \bfseries

        {\Large Московский авиационный институт\\ (национальный исследовательский университет)}
        
        \vspace{48pt}
        
        {\large Факультет информационных технологий и прикладной математики}
        
        \vspace{36pt}
        
        {\large Кафедра вычислительной математики и~программирования}
        
        \vspace{48pt}
        
        Курсовой проект по курсу <<Операционные системы>>

    \end{center}
    
    \vspace{140pt}
    
    \begin{flushright}
    \begin{tabular}{rl}
    Студент: & А.\,Р. Боташев \\
    Преподаватель: & Е.\,С. Миронов \\
    Группа: & М8О-201Б-21 \\
    Дата: & \\
    Оценка: & \\
    Подпись: & \\
    \end{tabular}
    \end{flushright}
    
    \vfill
    
    \begin{center}
    \bfseries
    Москва, \the\year
    \end{center}
\end{titlepage}
    
\pagebreak

\section{Постановка задачи}

Необходимо написать 3-и программы. Далее будем обозначать эти программы A, B, C.

Программа A принимает из стандартного потока ввода строки, а далее их отправляет программе С. Отправка строк должна производится построчно.Программа C печатает в стандартый вывод, полученную строку от программы A. После получения программа C отправляет программе А сообщение о том, что строка получена. До тех пор пока программа А не примет«сообщение о получение строки» от программы С, она не может отправялять следующую строку программе С.
Программа B пишет в стандартный вывод количество отправленных символов программой А и количество принятыхсимволов программой С. Данную информацию программа B получает от программ A и C соответственно

\section{Сведения о программе}

Программы написаны на языке C++ для Unix подобной операционной системы на базе ядра Linux.
Для связи между процессами используется pipe

\section{Общий метод и алгоритм решения}
Программа А создает два дочерних процесса B и C, затем считывает строки из стандартного потока ввода, после чего передает строки процессу C

Процесс С пишет полученную строку, подтвердив получение строки процессу А

А отправляет размер отправленной строки программе B

C отправляет размер полученной строки программе B


\section{Листинг программы}

{\large\textbf{A.cpp}}

\begin{lstlisting}[language=C++]
#include <stdio.h>
#include <unistd.h>
#include <stdlib.h>
#include <signal.h>
#include <string.h>

#include "../include/get_line.h"

int id1, id2;
int pipeAC[2];
int pipeAB[2];
int pipeCA[2];
int pipeCB[2];

void sig_handler(int signal) {
    kill(id1, SIGUSR1);
    kill(id2, SIGUSR1);

    close(pipeAC[0]);
    close(pipeAC[1]);
    close(pipeCA[0]);
    close(pipeCA[1]);
    close(pipeAB[0]);
    close(pipeAB[1]);
    close(pipeCB[0]);
    close(pipeCB[1]);
    exit(0);
}

int main(){

    if (signal(SIGINT, sig_handler) == SIG_ERR) {
        printf("[%d] ", getpid());
        perror("Error signal ");
        return -1;
    }
    pipe(pipeAC);
    pipe(pipeAB);
    pipe(pipeCA);
    pipe(pipeCB);


    id1 = fork();
    if (id1 == -1)
    {
        perror("fork1");
        exit(-1);
    }
    else if(id1 == 0)
    {
        char name[] = "./B";
        char pAB[3] = "";
        sprintf(pAB, "%d", pipeAB[0]);
        char pCB[3] = "";
        sprintf(pCB, "%d", pipeCB[0]);

        close(pipeAC[0]);
        close(pipeAC[1]);
        close(pipeCA[0]);
        close(pipeCA[1]);
        close(pipeAB[1]);
        close(pipeCB[1]);

        execl(name, name, pAB, pCB, NULL);
    }
    else
    {
        id2 = fork();
        if (id2 == -1)
        {
            perror("fork2");
            exit(-1);
        }
        else if(id2 == 0)
        {
            char name[] = "./C";
            char pAC[3] = "";
            sprintf(pAC, "%d", pipeAC[0]);
            char pCA[3] = "";
            sprintf(pCA, "%d", pipeCA[1]);
            char pCB[3] = "";
            sprintf(pCB, "%d", pipeCB[1]);

            close(pipeAC[1]);
            close(pipeCA[0]);
            close(pipeAB[0]);
            close(pipeAB[1]);
            close(pipeCB[0]);
            execl(name, name, pAC, pCA, pCB, NULL);
        }
        else
        {
            char* line = NULL;
            int size;
            while((size = get_line(&line, STDIN_FILENO)) != 0)
            {
                write(pipeAC[1], &size, sizeof(int));
                write(pipeAC[1], line, size*sizeof(char));

                int ok;
                read(pipeCA[0], &ok, sizeof(int));

                write(pipeAB[1], &size, sizeof(int));
            }
            free(line);

            kill(id1, SIGUSR1);
            kill(id2, SIGUSR1);
        }
    }


    close(pipeAC[0]);
    close(pipeAC[1]);
    close(pipeCA[0]);
    close(pipeCA[1]);
    close(pipeAB[0]);
    close(pipeAB[1]);
    close(pipeCB[0]);
    close(pipeCB[1]);


}
\end{lstlisting}

{\large\textbf{B.cpp}}

\begin{lstlisting}[language=C++]
#include <stdio.h>
#include <unistd.h>
#include <stdlib.h>
#include <signal.h>

int pipeAB;
int pipeCB;


void sig_handler(int signal) {
    close(pipeAB);
    close(pipeCB);
    exit(0);
}

int main(int argc, char *argv[]){
    if (signal(SIGUSR1, sig_handler) == SIG_ERR) {
        printf("[%d] ", getpid());
        perror("Error signal ");
        return -1;
    }
    pipeAB = atoi(argv[1]);
    pipeCB = atoi(argv[2]);

    int sizeA;
    int sizeB;
    while(read(pipeAB, &sizeA, sizeof(int)) > 0 && read(pipeCB, &sizeB, sizeof(int)) > 0){
        printf("[B] Get A: %d; Get C: %d\n", sizeA, sizeB);
    }

}
\end{lstlisting}

{\large\textbf{C.cpp}}

\begin{lstlisting}[language=C++]

#include <stdio.h>
#include <unistd.h>
#include <stdlib.h>
#include <signal.h>

int pipeAC;
int pipeCA;
int pipeCB;


void sig_handler(int signal) {
    close(pipeAC);
    close(pipeCA);
    close(pipeCB);

    exit(0);

}

int main(int argc, char *argv[]){
    if (signal(SIGUSR1, sig_handler) == SIG_ERR) {
        perror("[C] Error signal ");
        return -1;
    }
    pipeAC = atoi(argv[1]);
    pipeCA = atoi(argv[2]);
    pipeCB = atoi(argv[3]);

    int sizeA;
    while(read(pipeAC, &sizeA, sizeof(int))> 0){
        char line[sizeA+1];
        line[sizeA] = '\0';
        read(pipeAC, line, sizeA * sizeof(char));
        printf("[C] Get from A: %s\n", line);

        int ok = 1;
        write(pipeCA, &ok, sizeof(int));

        write(pipeCB, &sizeA, sizeof(int));

    }

}
\end{lstlisting}

\section{Демонстрация работы программ}

\begin{alltt}
botashev@botashev-laptop:~/ClionProjects/os_labs/course_proj$ ./A
hello
[C] Get from A: hello
[B] Get A: 5; Get C: 5
world
[C] Get from A: world
[B] Get A: 5; Get C: 5
qwertyuioplkmasnfdkja
[C] Get from A: qwertyuioplkmasnfdkja
[B] Get A: 21; Get C: 21
\                     f
[C] Get from A:                      f
[B] Get A: 22; Get C: 22
^C

\end{alltt}
\pagebreak

\section{Вывод}

Данная крусовая работа основывается на знаниях полученных в ходе изучения курса. По итогу мы получили нейколько программ, которые взаимодействуют друг с другом с помощью pipe.
Задача курсового проекта не сложна в реализации, но ее реализация обощает и закрепляет полученные в курсе знания.

\end{document}
