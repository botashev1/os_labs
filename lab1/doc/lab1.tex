\documentclass[pdf, unicode, 12pt, a4paper,oneside,fleqn]{article}
\usepackage{graphicx}
\graphicspath{{img/}}
\usepackage{log-style}
\begin{document}

\begin{titlepage}
    \begin{center}
        \bfseries

        {\Large Московский авиационный институт\\ (национальный исследовательский университет)}
        
        \vspace{48pt}
        
        {\large Факультет информационных технологий и прикладной математики}
        
        \vspace{36pt}
        
        {\large Кафедра вычислительной математики и~программирования}
        
        \vspace{48pt}
        
        Лабораторная работа \textnumero 1 по курсу \enquote{Операционные системы}

        \vspace{48pt}

        Приобретение практических навыков диагностики работы программного обеспечения.
    \end{center}
    
    \vspace{150pt}
    
    \begin{flushright}
    \begin{tabular}{rl}
    Студент: & А.\,Р. Боташев \\
    Преподаватель: & Е.\,С. Миронов \\
    Группа: & М8О-201Б-21 \\
    Дата: & \\
    Оценка: & \\
    Подпись: & \\
    \end{tabular}
    \end{flushright}
    
    \vfill
    
    \begin{center}
    \bfseries
    Москва, \the\year
    \end{center}
\end{titlepage}
    
\pagebreak

\section{Постановка задачи}

При выполнении последующих лабораторных работ необходимо продемонстрировать 
ключевые системные вызовы, которые в них используются.

Используемые утилиты: strace.

\section{Описание утилит}

Strace \--- утилита Linux для отслеживания системных вызовов, которые представляют собой
механизм трансляции, обеспечивающий интерфейс между процессом и ядром. Работы strace возможна
благодаря функции ядра ptrace. С помощью данной утилиты можно понять, что процесс пытается сделать в данное время.


\section{Пример использования утилиты}

{\large\textbf{Лабораторная работа 2}}

\begin{alltt}
\tiny
botashev@botashev-laptop:~/ClionProjects/os_labs/cmake-build-debug/lab2$ cat test.txt 
qwerty
botashev@botashev-laptop:~/ClionProjects/os_labs/cmake-build-debug/lab2$ strace -o str.log ./lab2 < test.txt
QWERTY
botashev@botashev-laptop:~/ClionProjects/os_labs/cmake-build-debug/lab2$ cat str.log 
execve("./lab2", ["./lab2"], 0x7ffd90d3b6b0 /* 56 vars */) = 0
brk(NULL)                               = 0x55b84caf7000
arch_prctl(0x3001 /* ARCH_??? */, 0x7fffa819a580) = -1 EINVAL (Недопустимый аргумент)
mmap(NULL, 8192, PROT_READ|PROT_WRITE, MAP_PRIVATE|MAP_ANONYMOUS, -1, 0) = 0x7fe95b6d4000
access("/etc/ld.so.preload", R_OK)      = -1 ENOENT (Нет такого файла или каталога)
openat(AT_FDCWD, "/etc/ld.so.cache", O_RDONLY|O_CLOEXEC) = 3
newfstatat(3, "", {st_mode=S_IFREG|0644, st_size=69279, ...}, AT_EMPTY_PATH) = 0
mmap(NULL, 69279, PROT_READ, MAP_PRIVATE, 3, 0) = 0x7fe95b6c3000
close(3)                                = 0
openat(AT_FDCWD, "/lib/x86_64-linux-gnu/libstdc++.so.6", O_RDONLY|O_CLOEXEC) = 3
read(3, "\177ELF\2\1\1\3\0\0\0\0\0\0\0\0\3\0>\0\1\0\0\0\0\0\0\0\0\0\0\0"..., 832) = 832
newfstatat(3, "", {st_mode=S_IFREG|0644, st_size=2252096, ...}, AT_EMPTY_PATH) = 0
mmap(NULL, 2267328, PROT_READ, MAP_PRIVATE|MAP_DENYWRITE, 3, 0) = 0x7fe95b499000
mmap(0x7fe95b533000, 1114112, PROT_READ|PROT_EXEC, MAP_PRIVATE|MAP_FIXED|MAP_DENYWRITE, 3, 0x9a000) = 0x7fe95b533000
mmap(0x7fe95b643000, 454656, PROT_READ, MAP_PRIVATE|MAP_FIXED|MAP_DENYWRITE, 3, 0x1aa000) = 0x7fe95b643000
mmap(0x7fe95b6b2000, 57344, PROT_READ|PROT_WRITE, MAP_PRIVATE|MAP_FIXED|MAP_DENYWRITE, 3, 0x218000) = 0x7fe95b6b2000
mmap(0x7fe95b6c0000, 10432, PROT_READ|PROT_WRITE, MAP_PRIVATE|MAP_FIXED|MAP_ANONYMOUS, -1, 0) = 0x7fe95b6c0000
close(3)                                = 0
openat(AT_FDCWD, "/lib/x86_64-linux-gnu/libgcc_s.so.1", O_RDONLY|O_CLOEXEC) = 3
read(3, "\177ELF\2\1\1\0\0\0\0\0\0\0\0\0\3\0>\0\1\0\0\0\0\0\0\0\0\0\0\0"..., 832) = 832
newfstatat(3, "", {st_mode=S_IFREG|0644, st_size=125488, ...}, AT_EMPTY_PATH) = 0
mmap(NULL, 127720, PROT_READ, MAP_PRIVATE|MAP_DENYWRITE, 3, 0) = 0x7fe95b479000
mmap(0x7fe95b47c000, 94208, PROT_READ|PROT_EXEC, MAP_PRIVATE|MAP_FIXED|MAP_DENYWRITE, 3, 0x3000) = 0x7fe95b47c000
mmap(0x7fe95b493000, 16384, PROT_READ, MAP_PRIVATE|MAP_FIXED|MAP_DENYWRITE, 3, 0x1a000) = 0x7fe95b493000
mmap(0x7fe95b497000, 8192, PROT_READ|PROT_WRITE, MAP_PRIVATE|MAP_FIXED|MAP_DENYWRITE, 3, 0x1d000) = 0x7fe95b497000
close(3)                                = 0
openat(AT_FDCWD, "/lib/x86_64-linux-gnu/libc.so.6", O_RDONLY|O_CLOEXEC) = 3
read(3, "\177ELF\2\1\1\3\0\0\0\0\0\0\0\0\3\0>\0\1\0\0\0P\237\2\0\0\0\0\0"..., 832) = 832
pread64(3, "\6\0\0\0\4\0\0\0@\0\0\0\0\0\0\0@\0\0\0\0\0\0\0@\0\0\0\0\0\0\0"..., 784, 64) = 784
pread64(3, "\4\0\0\0 \0\0\0\5\0\0\0GNU\0\2\0\0\300\4\0\0\0\3\0\0\0\0\0\0\0"..., 48, 848) = 48
pread64(3, "\4\0\0\0\24\0\0\0\3\0\0\0GNU\0i8\235HZ\227\223\333\350s\360\352,\223\340."..., 68, 896) = 68
newfstatat(3, "", {st_mode=S_IFREG|0644, st_size=2216304, ...}, AT_EMPTY_PATH) = 0
pread64(3, "\6\0\0\0\4\0\0\0@\0\0\0\0\0\0\0@\0\0\0\0\0\0\0@\0\0\0\0\0\0\0"..., 784, 64) = 784
mmap(NULL, 2260560, PROT_READ, MAP_PRIVATE|MAP_DENYWRITE, 3, 0) = 0x7fe95b251000
mmap(0x7fe95b279000, 1658880, PROT_READ|PROT_EXEC, MAP_PRIVATE|MAP_FIXED|MAP_DENYWRITE, 3, 0x28000) = 0x7fe95b279000
mmap(0x7fe95b40e000, 360448, PROT_READ, MAP_PRIVATE|MAP_FIXED|MAP_DENYWRITE, 3, 0x1bd000) = 0x7fe95b40e000
mmap(0x7fe95b466000, 24576, PROT_READ|PROT_WRITE, MAP_PRIVATE|MAP_FIXED|MAP_DENYWRITE, 3, 0x214000) = 0x7fe95b466000
mmap(0x7fe95b46c000, 52816, PROT_READ|PROT_WRITE, MAP_PRIVATE|MAP_FIXED|MAP_ANONYMOUS, -1, 0) = 0x7fe95b46c000
close(3)                                = 0
openat(AT_FDCWD, "/lib/x86_64-linux-gnu/libm.so.6", O_RDONLY|O_CLOEXEC) = 3
read(3, "\177ELF\2\1\1\3\0\0\0\0\0\0\0\0\3\0>\0\1\0\0\0\0\0\0\0\0\0\0\0"..., 832) = 832
newfstatat(3, "", {st_mode=S_IFREG|0644, st_size=940560, ...}, AT_EMPTY_PATH) = 0
mmap(NULL, 942344, PROT_READ, MAP_PRIVATE|MAP_DENYWRITE, 3, 0) = 0x7fe95b16a000
mmap(0x7fe95b178000, 507904, PROT_READ|PROT_EXEC, MAP_PRIVATE|MAP_FIXED|MAP_DENYWRITE, 3, 0xe000) = 0x7fe95b178000
mmap(0x7fe95b1f4000, 372736, PROT_READ, MAP_PRIVATE|MAP_FIXED|MAP_DENYWRITE, 3, 0x8a000) = 0x7fe95b1f4000
mmap(0x7fe95b24f000, 8192, PROT_READ|PROT_WRITE, MAP_PRIVATE|MAP_FIXED|MAP_DENYWRITE, 3, 0xe4000) = 0x7fe95b24f000
close(3)                                = 0
mmap(NULL, 8192, PROT_READ|PROT_WRITE, MAP_PRIVATE|MAP_ANONYMOUS, -1, 0) = 0x7fe95b168000
arch_prctl(ARCH_SET_FS, 0x7fe95b1693c0) = 0
set_tid_address(0x7fe95b169690)         = 10124
set_robust_list(0x7fe95b1696a0, 24)     = 0
rseq(0x7fe95b169d60, 0x20, 0, 0x53053053) = 0
mprotect(0x7fe95b466000, 16384, PROT_READ) = 0
mprotect(0x7fe95b24f000, 4096, PROT_READ) = 0
mprotect(0x7fe95b497000, 4096, PROT_READ) = 0
mmap(NULL, 8192, PROT_READ|PROT_WRITE, MAP_PRIVATE|MAP_ANONYMOUS, -1, 0) = 0x7fe95b166000
mprotect(0x7fe95b6b2000, 45056, PROT_READ) = 0
mprotect(0x55b84beae000, 4096, PROT_READ) = 0
mprotect(0x7fe95b70e000, 8192, PROT_READ) = 0
prlimit64(0, RLIMIT_STACK, NULL, {rlim_cur=8192*1024, rlim_max=RLIM64_INFINITY}) = 0
munmap(0x7fe95b6c3000, 69279)           = 0
getrandom("\x6c\x59\x8f\x28\x11\x32\xdb\x68", 8, GRND_NONBLOCK) = 8
brk(NULL)                               = 0x55b84caf7000
brk(0x55b84cb18000)                     = 0x55b84cb18000
futex(0x7fe95b6c077c, FUTEX_WAKE_PRIVATE, 2147483647) = 0
newfstatat(0, "", {st_mode=S_IFREG|0664, st_size=7, ...}, AT_EMPTY_PATH) = 0
read(0, "qwerty\n", 4096)               = 7
read(0, "", 4096)                       = 0
pipe2([3, 4], 0)                        = 0
pipe2([5, 6], 0)                        = 0
clone(child_stack=NULL, flags=CLONE_CHILD_CLEARTID|CLONE_CHILD_SETTID|SIGCHLD, child_tidptr=0x7fe95b169690) = 10125
close(3)                                = 0
write(4, "qwerty\n", 7)                 = 7
close(4)                                = 0
pipe2([3, 4], 0)                        = 0
clone(child_stack=NULL, flags=CLONE_CHILD_CLEARTID|CLONE_CHILD_SETTID|SIGCHLD, child_tidptr=0x7fe95b169690) = 10126
close(4)                                = 0
close(6)                                = 0
close(5)                                = 0
wait4(-1, NULL, 0, NULL)                = 10125
--- SIGCHLD {si_signo=SIGCHLD, si_code=CLD_EXITED, si_pid=10125, si_uid=1000, si_status=0, si_utime=0, si_stime=0} ---
read(3, "Q", 1)                         = 1
read(3, "W", 1)                         = 1
read(3, "E", 1)                         = 1
read(3, "R", 1)                         = 1
read(3, "T", 1)                         = 1
read(3, "Y", 1)                         = 1
read(3, "\n", 1)                        = 1
close(3)                                = 0
newfstatat(1, "", {st_mode=S_IFCHR|0620, st_rdev=makedev(0x88, 0), ...}, AT_EMPTY_PATH) = 0
write(1, "QWERTY\n", 7)                 = 7
exit_group(0)                           = ?
+++ exited with 0 +++
\end{alltt}

\end{verbatim}
}

\pagebreak

\section{Вывод}

Strace \--- удобный инструмент для системных вызовов, который полезен при отладке и тестировании программ.
Несмотря на то, что протоколы по умолчанию кажутся слишком объемными, информацию в 
них можно отфильтровать с помощью ключей.

\end{document}
